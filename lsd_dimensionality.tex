\documentclass[11pt,a4paper]{article}
\usepackage[utf8]{inputenc}
\usepackage{amsmath,amssymb}
\usepackage{graphicx}
\usepackage[margin=1in]{geometry}
\usepackage{hyperref}
\usepackage{natbib}
\usepackage{booktabs}

\title{Do Psychedelics Increase Cortical Dimensionality? \\
\large A Question We Cannot Yet Answer}

\author{Ian Todd\\
\texttt{itod2305@uni.sydney.edu.au}}

\date{\today}

\begin{document}

\maketitle

\begin{abstract}
If the cortex operates as a system of coupled neural oscillators, it is inherently high-dimensional. A natural question arises: do psychedelics change this dimensionality? We argue that 5-HT2A agonism should increase the number of independent oscillatory modes by reducing synchronization constraints. We attempted to test this using MEG and fMRI data from psychedelic studies. MEG shows that classical psychedelics produce oscillatory desynchronization (psilocybin: $-15\%$, $p = 0.003$) while ketamine does not---consistent with the hypothesis. However, fMRI-derived dimensionality metrics show null effects ($N = 7$, $p = 0.47$). We conclude that while the question is well-posed and the prediction is clear, current neuroimaging methods cannot answer it. The core problem is that ``dimensionality'' as measured by participation ratio may not correspond to ``dimensionality'' as theoretically defined. We outline what would be needed to resolve this.
\end{abstract}

\section{The Question}

Recent work suggests that cortical computation relies fundamentally on oscillatory dynamics \citep{miller2018working, buzsaki2006rhythms}. If this is correct, the cortex is a high-dimensional dynamical system: the state space includes the phases and amplitudes of oscillators across frequencies and regions.

This raises a simple question: \textbf{do psychedelics change the dimensionality of this system?}

By ``dimensionality'' we mean the number of independent modes of variation---how many degrees of freedom are actively being used. A system with 1000 oscillators might have effective dimensionality of 10 (if they're all synchronized) or 500 (if they're largely independent).

The question matters because dimensionality determines computational capacity. Low-dimensional systems are constrained to a narrow manifold; high-dimensional systems can explore more configurations. If psychedelics increase dimensionality, this could explain both their acute effects (access to unusual states) and their therapeutic potential (escape from rigid attractor patterns).

\section{The Prediction}

Classical psychedelics are 5-HT2A agonists. The 5-HT2A receptor is concentrated on layer 5 pyramidal neurons, particularly in apical dendrites \citep{nichols2016psychedelics}. Activation increases dendritic gain---the same input produces larger postsynaptic responses.

In a coupled oscillator system, this predicts desynchronization:
\begin{itemize}
\item Higher gain $\rightarrow$ neurons respond more to local inputs
\item Local inputs include noise and weak signals normally below threshold
\item This disrupts the coherent oscillations that synchronize populations
\item More independent oscillators $\rightarrow$ higher effective dimensionality
\end{itemize}

The prediction is qualitative: psychedelics should \emph{increase} dimensionality. We have no theoretical basis to predict the magnitude.

\section{What We Tried}

\subsection{MEG Analysis}

We analyzed MEG data from 136 sessions across four compounds: psilocybin ($N=40$), LSD ($N=30$), ketamine ($N=36$), and tiagabine ($N=30$). Data were from publicly available datasets \citep{muthukumaraswamy2013broadband, carhart2016neural}.

MEG measures synchronous postsynaptic currents. We computed the participation ratio of sensor covariance as a proxy for oscillatory coherence structure. Lower values indicate more distributed, desynchronized activity.

\textbf{Results:}
\begin{table}[h]
\centering
\begin{tabular}{lcccc}
\toprule
Compound & Change & $p$ & Cohen's $d$ \\
\midrule
Psilocybin & $-15.0\%$ & 0.003 & $-0.78$ \\
LSD & $-13.4\%$ & 0.082 & $-0.50$ \\
Ketamine & $+5.7\%$ & 0.290 & $+0.26$ \\
Tiagabine & $+10.8\%$ & 0.307 & $+0.28$ \\
\bottomrule
\end{tabular}
\end{table}

Classical psychedelics (5-HT2A agonists) show significant desynchronization. Ketamine (NMDA antagonist) does not. This is consistent with the prediction---but desynchronization is not the same as dimensionality increase. The number of independent modes could stay constant even as synchronization decreases.

\subsection{fMRI Analysis}

We analyzed fMRI data from the Siegel psilocybin precision functional mapping study (OpenNeuro ds006072): 7 subjects, 124 sessions total, with dense baseline imaging \citep{siegel2025psilocybin}.

We computed effective dimensionality ($D_{\mathrm{eff}}$) via participation ratio from CIFTI grayordinate time series (91,206 voxels):
\begin{equation}
D_{\mathrm{eff}} = \frac{\left(\sum_i \lambda_i\right)^2}{\sum_i \lambda_i^2}
\end{equation}

\textbf{Results:}
\begin{itemize}
\item Baseline: $D_{\mathrm{eff}} = 51.5 \pm 7.4$
\item Drug: $D_{\mathrm{eff}} = 49.0 \pm 12.4$
\item Change: $-5.7\%$ (wrong direction)
\item Statistics: $t(6) = -0.78$, $p = 0.47$, $d = -0.32$
\end{itemize}

The result is null. Two subjects increased, five decreased. Within-subject baseline variability spanned a threefold range (25--75), dwarfing any plausible drug effect.

\section{Why We Can't Answer the Question}

The MEG and fMRI results appear contradictory: MEG shows desynchronization, fMRI shows no dimensionality change. Several interpretations are possible:

\textbf{1. They measure different things.} MEG detects synchronous currents; fMRI detects metabolic demand. Desynchronization could occur without changing the diversity of hemodynamic patterns.

\textbf{2. Temporal resolution mismatch.} BOLD integrates over $\sim$10 seconds. Rapid dimensionality fluctuations may be smoothed away.

\textbf{3. The fMRI effect exists but is undetectable.} With baseline variability spanning 25--75 and only 7 subjects, statistical power is severely limited.

\textbf{4. The participation ratio doesn't measure ``true'' dimensionality.} This is the core problem.

\subsection{The Measurement Problem}

``Dimensionality'' as theoretically defined is the number of independent degrees of freedom in the underlying dynamical system. ``Dimensionality'' as measured by participation ratio is a property of the eigenspectrum of the observed covariance matrix.

These are not the same thing. The relationship depends on:
\begin{itemize}
\item The measurement modality (MEG, fMRI, ECoG, etc.)
\item The spatial sampling (parcellation, electrode placement)
\item The temporal sampling (TR, epoch length)
\item Preprocessing choices (filtering, motion correction)
\item Noise properties of the recording
\end{itemize}

Different choices produce different numbers. We have no ground truth to calibrate against.

\subsection{The Effect Size Problem}

Even if participation ratio perfectly tracked true dimensionality, we have no theoretical basis to predict how much it should change. The theory says ``psychedelics increase dimensionality''---it doesn't say by how much.

This makes falsification difficult. A null result could mean:
\begin{itemize}
\item The effect doesn't exist
\item The effect exists but is small
\item The effect exists but our measure doesn't capture it
\item The effect exists but noise overwhelms it
\end{itemize}

We cannot distinguish these possibilities.

\section{What Would Be Needed}

To answer the question ``do psychedelics increase cortical dimensionality?'' we would need:

\textbf{1. A validated dimensionality measure.} This would require:
\begin{itemize}
\item A system with known ground-truth dimensionality
\item Demonstration that the measure tracks true dimensionality across conditions
\item Understanding of how measurement parameters affect the estimate
\end{itemize}

\textbf{2. Sufficient statistical power.} Given observed variability ($\sigma \approx 15$), detecting a 10\% effect would require $N > 50$ subjects.

\textbf{3. Or, a qualitative signature.} Instead of asking ``does $D_{\mathrm{eff}}$ increase?'' we might ask ``does the eigenspectrum shape change in a characteristic way?'' This would be more robust to calibration issues.

\section{What We Can Say}

\begin{enumerate}
\item \textbf{The question is well-posed.} If cortex is oscillator-based, dimensionality is a meaningful concept.

\item \textbf{The prediction is clear.} 5-HT2A gain increase should desynchronize oscillators, increasing independent modes.

\item \textbf{MEG shows desynchronization.} Classical psychedelics reduce oscillatory coherence; ketamine does not. This is mechanism-specific.

\item \textbf{fMRI shows nothing.} Participation ratio in BOLD data does not change detectably with psilocybin.

\item \textbf{We cannot conclude whether dimensionality increases.} The measurement tools are not adequate to the question.
\end{enumerate}

\section{Why This Matters}

The clinical efficacy of psychedelics is increasingly established. The mechanism is not. ``Dimensionality expansion'' is an attractive hypothesis because it provides a computational account: psychedelics temporarily increase the brain's configuration space, enabling escape from pathological attractors.

But attractiveness is not evidence. The hypothesis remains untested---not because we haven't tried, but because we lack the measurement technology to test it. This is an honest statement of the current situation.

Future progress requires either better measurement methods or stronger theoretical predictions that can be tested with existing tools. Until then, the question ``do psychedelics increase cortical dimensionality?'' remains open.

\section*{Acknowledgements}

The author thanks the original investigators who made their datasets publicly available on OpenNeuro.

\section*{Data Availability}

All data analyzed are publicly available: LSD dataset (ds003059) and psilocybin dataset (ds006072) at \url{https://openneuro.org}.

\section*{Code Availability}

Analysis code is available at \url{https://github.com/todd866/lsd-dimensionality}.

\bibliographystyle{unsrtnat}
\bibliography{references}

\end{document}
