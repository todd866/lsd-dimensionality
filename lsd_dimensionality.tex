\documentclass[11pt,a4paper]{article}
\usepackage[utf8]{inputenc}
\usepackage{amsmath,amssymb}
\usepackage{graphicx}
\usepackage[margin=1in]{geometry}
\usepackage{hyperref}
\usepackage{natbib}
\usepackage{booktabs}
\usepackage{xcolor}

\title{Psychedelics as Dimensionality Modulators: \\
A Cortical Reservoir Theory of Serotonergic Plasticity}

\author{Ian Todd\\
Sydney Medical School, University of Sydney\\
\texttt{itod2305@uni.sydney.edu.au}}

\date{\today}

\begin{document}

\maketitle

\begin{abstract}
Classical psychedelics produce profound alterations in perception, cognition, and sense of self. Despite decades of research, a unifying computational framework for these effects remains elusive. Here we propose that the primary action of 5-HT2A agonists is to \textbf{modulate the effective dimensionality} of cortical dynamics---the number of independent modes available to the cortical reservoir. Through dendritic gain amplification in layer 5 pyramidal neurons, psychedelics expand the eigenmode spectrum of cortical oscillator fields, enabling the system to explore configurations inaccessible under baseline conditions. This dimensionality expansion is inherently transient: sustained 5-HT2A activation triggers receptor downregulation and homeostatic compression, followed by a \textbf{recanalization phase} during which the system reorganizes onto a modified---but normal-dimensional---attractor landscape. We formalize this three-phase model (overshoot $\rightarrow$ refractory $\rightarrow$ recanalization) and show how it unifies acute phenomenology, therapeutic afterglow effects, and long-term plasticity. Crucially, within-subjects pharmacological control reveals that dimensionality expansion is specific to 5-HT2A agonism: psilocybin produces robust $D_{\mathrm{eff}}$ increases while the active control methylphenidate (a dopamine/norepinephrine reuptake inhibitor) produces dimensional \emph{compression}, strongly supporting mechanism specificity. The framework connects psychedelic neuroscience to broader principles of cortical computation, suggesting that dimensionality---not any specific neurotransmitter system---is the fundamental variable that psychedelics modulate. We propose Brain Rate Variability (BRV), the neural analogue of heart rate variability, as a clinically accessible biomarker for tracking dimensionality dynamics across the psychedelic arc, and discuss implications for precision dosing, combination therapies, and risk stratification.
\end{abstract}

\textbf{Keywords:} psychedelics; effective dimensionality; 5-HT2A; reservoir computing; neural plasticity; LSD; psilocybin; cortical dynamics; brain rate variability

\section{Introduction}

The resurgence of psychedelic research represents one of the most significant developments in psychiatry and neuroscience of the past decade. Clinical trials have demonstrated remarkable efficacy: psilocybin shows robust effects for treatment-resistant depression \citep{carhart2016psilocybin, davis2021effects, goodwin2022single}, MDMA-assisted therapy produces breakthrough results for PTSD \citep{mitchell2021mdma, mithoefer2019mdma}, and growing evidence supports therapeutic applications for addiction \citep{bogenschutz2015psilocybin, johnson2014pilot}, anxiety in terminal illness \citep{griffiths2016psilocybin, grob2011pilot}, and obsessive-compulsive disorder \citep{moreno2006safety}. LSD microdosing, though less rigorously studied, shows promise for mood enhancement and cognitive flexibility \citep{fadiman2011psychedelic, hutten2020mood, prochazkova2018exploring}.

This clinical momentum has been matched by unprecedented neuroimaging data sharing. Public repositories now host multiple high-quality psychedelic datasets: the Carhart-Harris LSD dataset (OpenNeuro ds003059) provides within-subjects BOLD fMRI under 75$\mu$g IV LSD versus placebo \citep{carhart2016neural}, while the recent Siegel precision functional mapping study (OpenNeuro ds006072) offers dense longitudinal imaging across psilocybin and methylphenidate sessions with preprocessed CIFTI surface data \citep{siegel2025psilocybin}. Additional datasets covering ayahuasca, DMT, and ketamine are increasingly available, enabling rigorous replication and cross-compound meta-analysis. This data ecosystem transforms psychedelic neuroscience from isolated studies into a cumulative science capable of testing mechanistic theories across compounds, doses, and populations.

Yet despite this clinical progress, a fundamental question remains: what are psychedelics actually \emph{doing} to the brain? Current frameworks emphasize specific receptor pharmacology, network connectivity changes, or entropic brain dynamics. While each captures important aspects of the psychedelic state, none provides a unified computational account that explains:
\begin{itemize}
\item Why acute effects are so profoundly different from baseline consciousness
\item Why therapeutic benefits often emerge \emph{after} the acute experience ends
\item Why these compounds produce lasting plasticity from single or few doses
\item Why set and setting matter so dramatically for outcomes
\item Why tolerance develops rapidly but sensitization can occur with spacing
\item Why the same compound produces radically different experiences across individuals
\end{itemize}

Here we propose that psychedelics are fundamentally \textbf{dimensionality modulators}---they alter the number of independent dynamical modes available to cortical computation. This framework unifies disparate observations across scales from receptor pharmacology to phenomenology, and makes specific, testable predictions about the neural mechanisms underlying both acute effects and therapeutic outcomes.

\subsection{The Entropic Brain Hypothesis and Its Limitations}

The most influential computational framework for psychedelics is the Entropic Brain Hypothesis (EBH), proposed by \citet{carhart2014entropic} and elaborated in subsequent work \citep{carhart2018entropic, carhart2017relaxed}. The EBH posits that psychedelics increase the entropy of spontaneous brain activity, relaxing the normally constrained dynamics and enabling exploration of a broader state space.

The EBH has substantial empirical support. Psychedelics reliably increase measures of neural entropy and signal diversity \citep{schartner2017increased, timmermann2019neural}, flatten the cortical hierarchy \citep{tagliazucchi2016increased}, and dissolve the structured activity of the default mode network (DMN) \citep{carhart2012neural, palhano2015default}. The REBUS (Relaxed Beliefs Under Psychedelics) extension \citep{carhart2017relaxed} connects these entropic changes to predictive processing frameworks, suggesting that psychedelics relax the precision-weighting of prior beliefs.

However, the EBH faces several limitations. First, ``entropy'' is a broad concept that conflates multiple distinct phenomena---signal complexity, unpredictability, and state space exploration are not equivalent \citep{mediano2020measuring}. Second, the relationship between neural entropy and therapeutic outcome is unclear; some highly entropic states (seizures, delirium) are profoundly pathological. Third, the EBH does not explain the temporal dynamics of the psychedelic experience---why entropy increases acutely, why tolerance develops, and why lasting changes emerge after the acute state resolves.

We propose that \textbf{effective dimensionality} provides a more precise and mechanistically grounded framework than entropy. Dimensionality captures the computational essence of what entropy measures---the richness of the dynamical repertoire---while connecting directly to neural circuit mechanisms and making quantitative predictions about scaling and limits.

Recent work has begun applying dimensionality metrics to psychedelic neuroimaging. \citet{moujaes2024ketamine} used the participation ratio to compare connectivity signatures across ketamine, LSD, and psilocybin, finding that ketamine produces higher-dimensional patterns than the classical serotonergic psychedelics. However, their analysis treats dimensionality as a \emph{descriptive metric} for drug fingerprinting rather than as the mechanistic target of therapeutic action. Our framework differs fundamentally: we propose that dimensionality expansion is not merely a correlate of the psychedelic state but its \emph{computational function}---the means by which psychedelics enable exploration of off-manifold configurations. Critically, our three-phase model (overshoot $\rightarrow$ refractory $\rightarrow$ recanalization) explains why therapeutic benefits persist after dimensionality returns to baseline, a temporal dynamic that purely acute analyses cannot address.

\subsection{Effective Dimensionality as a Cortical State Variable}

The concept of effective dimensionality ($D_{\mathrm{eff}}$) captures how many independent degrees of freedom are actually being utilized by a dynamical system \citep{cunningham2014dimensionality, gao2017theory}. For cortical networks, $D_{\mathrm{eff}}$ reflects the number of eigenmode directions along which neural population activity has substantial variance. The participation ratio provides a standard measure:
\begin{equation}
D_{\mathrm{eff}} = \frac{\left(\sum_i \lambda_i\right)^2}{\sum_i \lambda_i^2}
\end{equation}
where $\lambda_i$ are eigenvalues of the covariance matrix of neural activity.

Under baseline conditions, cortical dynamics occupy a surprisingly low-dimensional manifold despite the astronomical number of potential configurations \citep{gallego2017neural, jazayeri2021interpreting, stringer2019high}. Motor cortex activity during reaching lies on manifolds of dimension 10-20, not the thousands one might expect from the number of neurons \citep{churchland2012neural, kaufman2014cortical}. Visual cortex responses, despite their complexity, can be captured by relatively few principal components \citep{stringer2019spontaneous}. Even ``spontaneous'' resting activity shows strong dimensional constraints \citep{luczak2009spontaneous, miller2014visual}.

This dimensional constraint is not a limitation---it is the computational strategy. By confining dynamics to a learned subspace, the cortex achieves:
\begin{itemize}
\item \textbf{Noise robustness:} Activity orthogonal to the manifold is noise, automatically filtered \citep{kaufman2014cortical}
\item \textbf{Efficient readout:} Downstream areas need only monitor a low-dimensional projection \citep{sadtler2014neural}
\item \textbf{Fast learning:} New skills are acquired within existing subspaces when possible \citep{golub2018learning, sadtler2014neural}
\item \textbf{Stable memory:} Attractors in a constrained manifold are more robust \citep{chaudhuri2016computational}
\end{itemize}

However, dimensional constraint has a cost: it limits the space of reachable configurations. A system locked into a narrow manifold cannot explore radically different solutions. Motor cortex constrained to a 10-dimensional manifold cannot spontaneously discover a 50-dimensional movement strategy, even if that strategy would be superior \citep{sadtler2014neural}. This constraint-flexibility tradeoff is fundamental to neural computation.

This is precisely where psychedelics enter: they temporarily expand the accessible dimensionality, enabling exploration of configurations that are normally off-manifold.

\subsection{The Reservoir Computing Perspective}

Reservoir computing provides a natural theoretical framework for understanding cortical dimensionality \citep{jaeger2001echo, maass2002real, tanaka2019recent}. In this view, cortical networks function as high-dimensional nonlinear ``reservoirs'' that:
\begin{enumerate}
\item Receive low-dimensional inputs (sensory streams, internal goals)
\item Project these inputs into a high-dimensional dynamical space
\item Generate outputs via linear readout from the expanded representation
\end{enumerate}

The key insight is that reservoir computing power scales with the \emph{number of separable dynamical modes}---precisely what $D_{\mathrm{eff}}$ measures. A reservoir with higher effective dimensionality can separate more input patterns, support more complex nonlinear computations, and maintain longer memory traces \citep{legenstein2007edge, verstraeten2007experimental, lukosevicius2009reservoir}.

The ``edge of chaos'' literature demonstrates that computational capacity is maximized when reservoirs operate near a critical transition between ordered and chaotic dynamics \citep{bertschinger2004real, legenstein2007edge}. At this edge, dimensionality is high but not maximal---the system explores broadly while maintaining enough structure for reliable readout.

From this perspective, psychedelics do something remarkable: they \emph{temporarily increase reservoir capacity} by expanding the eigenmode spectrum. The acute state provides access to configurations that are normally off-manifold, enabling the system to explore solutions that would otherwise be unreachable. This is not merely ``adding noise'' (which would degrade computation) but systematically lowering activation thresholds for latent eigenmodes.

\section{Mechanism: 5-HT2A and Dendritic Gain}

Classical psychedelics---LSD, psilocybin, DMT, mescaline---share a common mechanism: agonism at the serotonin 5-HT2A receptor \citep{nichols2016psychedelics, vollenweider2010neurobiology}. While these compounds have additional pharmacological targets (5-HT2C, 5-HT1A, dopamine receptors), the 5-HT2A receptor is necessary and likely sufficient for the characteristic psychedelic effects \citep{preller2018neural, kometer2013activation, kraehenmann2017dreamlike}. Blocking 5-HT2A with ketanserin eliminates subjective effects and normalizes neural signatures \citep{preller2018neural, vollenweider1998effects}.

\subsection{Layer 5 Pyramidal Neurons as Cortical Amplifiers}

The 5-HT2A receptor is densely expressed on apical dendrites of layer 5 pyramidal neurons (L5PNs)---the primary output neurons of neocortex \citep{jakab1998receptor, weber2010localization, watakabe2009enriched}. This localization is functionally significant: apical dendrites integrate top-down contextual inputs and gate the influence of these inputs on neural output \citep{larkum2013cellular, larkum1999new}.

When activated by psychedelics, 5-HT2A signaling produces a constellation of electrophysiological effects:
\begin{itemize}
\item \textbf{Reduced afterhyperpolarization:} 5-HT2A activation reduces the slow afterhyperpolarizing current (sAHP), increasing neuronal excitability \citep{aghajanian1999serotonin, zhang2002serotonin}
\item \textbf{Enhanced calcium plateaus:} Dendritic calcium plateau potentials are facilitated, lowering the threshold for dendritic spikes \citep{andrade2011serotonergic, beique2007serotonin}
\item \textbf{Facilitated backpropagation:} Backpropagating action potentials reach further into the dendritic tree \citep{andrade2011serotonergic}
\item \textbf{Increased spontaneous EPSPs:} Glutamate release probability increases, elevating baseline excitatory drive \citep{marek1998frontocortical, aghajanian1997serotonin}
\item \textbf{Enhanced NMDA currents:} NMDA receptor function is potentiated, amplifying coincidence detection \citep{bhattacharyya2022serotonergic}
\end{itemize}

The net effect is \textbf{dendritic gain amplification}: inputs that would normally fail to drive somatic output now succeed. Weak, subthreshold patterns of synaptic input can trigger dendritic spikes and somatic action potentials. This is functionally equivalent to lowering the activation threshold for cortical response patterns---modes that are normally latent become active.

\subsection{Eigenmode Expansion in Cortical Networks}

\begin{figure}[htbp]
\centering
\includegraphics[width=\textwidth]{figures/fig1_eigenmode_mechanism.pdf}
\caption{\textbf{Eigenmode Expansion Mechanism.} (A) Eigenvalue spectra showing how psychedelic-state dynamics (red) maintain higher variance across more eigenmodes than baseline (blue), increasing the number of modes above activation threshold. (B) Participation ratio calculation: baseline dynamics concentrate variance on few modes ($D_{\mathrm{eff}} \approx 3$) while psychedelic dynamics distribute across many ($D_{\mathrm{eff}} \approx 8$). (C) Dendritic mechanism: 5-HT2A receptors on apical dendrites of layer 5 pyramidal neurons reduce afterhyperpolarization and enhance calcium spikes, effectively lowering the activation threshold for cortical response patterns.}
\label{fig:eigenmode}
\end{figure}

How does dendritic gain amplification translate to increased effective dimensionality at the network level? Cortical dynamics can be modeled as coupled oscillator fields where each oscillator represents the activity of a local neural population \citep{breakspear2010generative, cabral2014exploring, deco2017dynamics}. The effective dimensionality of this field depends on:
\begin{enumerate}
\item The number of distinct oscillator frequencies (frequency dispersion)
\item The strength of coupling between oscillators (synchronization tendency)
\item The noise level and intrinsic variability (stochastic mode activation)
\item The nonlinear activation thresholds (eigenmode accessibility)
\end{enumerate}

5-HT2A activation affects all four factors in ways that expand $D_{\mathrm{eff}}$:

\textbf{Frequency dispersion increases.} Psychedelics desynchronize cortical rhythms, particularly in the alpha band (8-12 Hz) \citep{muthukumaraswamy2013broadband, carhart2016neural}. This desynchronization reflects a broadening of the active frequency spectrum---more oscillatory modes with different frequencies become simultaneously active, increasing the dimensionality of the dynamical repertoire.

\textbf{Long-range coupling decreases.} DMN dissolution and reduced functional connectivity between distant regions \citep{carhart2012neural, tagliazucchi2016increased} indicate weakened long-range coupling. When coupling is strong, distant regions lock into coherent patterns, reducing independent degrees of freedom. Weakened coupling allows regions to explore more independently, increasing overall dimensionality.

\textbf{Spontaneous variability increases.} Enhanced spontaneous EPSPs and reduced sAHP increase intrinsic neural fluctuations \citep{marek1998frontocortical}. These fluctuations stochastically activate modes that would otherwise remain quiescent, expanding the explored configuration space.

\textbf{Activation thresholds decrease.} This is the direct effect of dendritic gain amplification. Eigenmodes of cortical dynamics that normally require strong, coordinated input to activate become accessible to weaker, more varied input patterns.

The combined effect is substantial expansion of $D_{\mathrm{eff}}$. The desynchronized, decoupled, variable state has more active eigenmodes than the synchronized, coupled, constrained baseline.

\subsection{The Ephaptic Dimension}

Beyond synaptic transmission, cortical neurons interact via ephaptic coupling---extracellular electric field effects that modulate neighboring neurons without synaptic contact \citep{anastassiou2011ephaptic, anastassiou2015biophysics, martinez2019ephaptic}. During synchronized oscillatory activity, coherent population rhythms generate substantial extracellular fields (1-5 mV/mm) that can shift neuronal membrane potentials by several millivolts \citep{frohlich2010endogenous, herreras2016local}.

Ephaptic coupling effectively creates a ``mean field'' constraint that tends to synchronize neighboring neurons. This constraint reduces effective dimensionality by forcing local populations into coherent states. The strength of ephaptic coupling scales with oscillatory power and coherence \citep{anastassiou2015biophysics}.

Psychedelic-induced desynchronization reduces ephaptic coupling strength by fragmenting the coherent population oscillations that generate strong extracellular fields. This releases neurons from a form of collective constraint, contributing to $D_{\mathrm{eff}}$ increase via a non-synaptic pathway.

The ephaptic contribution may explain why psychedelic effects are particularly prominent for alpha oscillations, which generate the largest extracellular fields due to their coherent, high-amplitude nature \citep{lopes2017visual}. Alpha suppression under psychedelics \citep{muthukumaraswamy2013broadband, carhart2016neural} may reflect not just reduced oscillatory drive but reduced ephaptic synchronization.

\subsection{Structural Plasticity and Dendritic Remodeling}

Recent work has revealed that psychedelics induce rapid structural plasticity in cortical neurons. A single dose of psilocybin, LSD, or DMT increases dendritic spine density and dendritic arbor complexity in prefrontal cortex within 24 hours \citep{ly2018psychedelics, shao2021psilocybin}. These changes are 5-HT2A-dependent and correlate with behavioral effects.

From our framework, structural plasticity represents the physical substrate of lasting dimensionality changes. Increased spine density provides more synaptic inputs, potentially enabling access to eigenmodes that were previously unreachable. Dendritic arbor expansion increases the integration volume for top-down inputs, amplifying the gain effects we have described.

Importantly, structural plasticity occurs during the acute phase but persists into the recanalization period. This provides a mechanism for how a transient dimensionality expansion can produce lasting reorganization: the expanded connectivity remains even after pharmacological effects resolve, supporting a modified attractor landscape.

\section{The Three-Phase Model}

We propose that the full psychedelic arc comprises three distinct phases characterized by different dimensionality regimes (Figure \ref{fig:three-phase}):

\begin{figure}[htbp]
\centering
\includegraphics[width=\textwidth]{figures/fig2_three_phase_model.pdf}
\caption{\textbf{The Three-Phase Model of Psychedelic Dimensionality Dynamics.} Effective dimensionality ($D_{\mathrm{eff}}$) follows a characteristic arc: Phase 1 (Overshoot) shows dramatic expansion above baseline during the acute psychedelic experience, driven by 5-HT2A activation and dendritic gain amplification. Phase 2 (Refractory) shows below-baseline compression due to receptor downregulation and signaling depletion. Phase 3 (Recanalization) shows return to baseline dimensionality but on a reorganized attractor landscape supported by structural plasticity.}
\label{fig:three-phase}
\end{figure}

\subsection{Phase 1: Overshoot ($D_{\mathrm{eff}} \gg D_{\mathrm{baseline}}$)}

The acute psychedelic state is characterized by dramatic dimensionality expansion. 5-HT2A activation amplifies dendritic gain, expands the eigenmode spectrum, and enables exploration of off-manifold configurations.

\textbf{Phenomenology:} The subjective effects of Phase 1 directly reflect expanded dimensionality:
\begin{itemize}
\item \textbf{Perceptual intensification:} More visual features and patterns are simultaneously represented, producing enhanced color, texture, and geometric complexity \citep{kometer2011psilocybin}
\item \textbf{Ego dissolution:} The narrative self, normally maintained by constrained DMN dynamics, fragments as self-referential processing loses its coherent attractor \citep{nour2016ego, milliere2017psychedelics}
\item \textbf{Time dilation:} Temporal integration, which relies on dimensional compression, becomes disrupted, producing subjective time expansion \citep{wittmann2015modulations, yanakieva2019effects}
\item \textbf{Novel associations:} Semantic and conceptual representations that are normally separated become accessible in the same activation space, enabling unusual connections \citep{family2016semantic, mason2021spontaneous}
\item \textbf{Synesthesia-like experiences:} Cross-modal representations become co-active as normally isolated sensory eigenmodes overlap \citep{sinke2012genuine, luke2012psychedelics}
\item \textbf{Mystical experience:} Boundary dissolution and unity experiences may reflect the loss of categorical distinctions that normally separate self from world \citep{barrett2015classic, griffiths2006psilocybin}
\end{itemize}

\textbf{Neural signatures:}
\begin{itemize}
\item Increased Lempel-Ziv complexity and neural entropy \citep{schartner2017increased, timmermann2019neural}
\item Alpha power suppression and desynchronization \citep{muthukumaraswamy2013broadband, carhart2016neural}
\item DMN dissolution and reduced hierarchical organization \citep{carhart2012neural, tagliazucchi2016increased}
\item Increased global functional connectivity diversity \citep{tagliazucchi2014enhanced}
\item Enhanced repertoire of functional connectivity states \citep{lord2019dynamical}
\end{itemize}

\textbf{Duration:} 4--8 hours for LSD, 4--6 hours for psilocybin, 15--30 minutes for DMT.

\textbf{Mechanism:} 5-HT2A activation $\rightarrow$ dendritic gain increase $\rightarrow$ lowered eigenmode thresholds $\rightarrow$ expanded $D_{\mathrm{eff}}$.

\subsection{Phase 2: Refractory Collapse ($D_{\mathrm{eff}} < D_{\mathrm{baseline}}$)}

Sustained 5-HT2A activation triggers homeostatic responses. The receptor undergoes rapid internalization and downregulation via $\beta$-arrestin-mediated endocytosis \citep{berry2017rapid, burt2018transcriptomic, gray2003cell}. Signaling intermediates (PLC, PKC, intracellular calcium stores) become depleted. As the molecular tide recedes, the system enters a refractory state characterized by \emph{lower-than-baseline} dimensionality.

\textbf{Phenomenology:} Phase 2 experiences reflect dimensional compression:
\begin{itemize}
\item \textbf{Cognitive fatigue:} Reduced processing capacity, difficulty with complex thought
\item \textbf{Emotional sensitivity:} Heightened reactivity as emotional regulation circuits are depleted
\item \textbf{Heightened suggestibility:} Reduced critical faculties, increased openness to influence
\item \textbf{Integration focus:} Natural tendency toward meaning-making and narrative construction
\item \textbf{Sleep disturbance:} Altered sleep architecture reflecting continued neurochemical perturbation
\end{itemize}

\textbf{Neural signatures:} Limited direct evidence, but predicted signatures include:
\begin{itemize}
\item Below-baseline entropy and complexity measures
\item Increased alpha coherence (rebound synchronization)
\item Temporarily reduced functional connectivity flexibility
\item PET evidence of reduced 5-HT2A availability
\end{itemize}

\textbf{Duration:} 1--7 days post-experience. This corresponds to the period of acute tolerance where re-dosing produces attenuated effects \citep{nichols2004hallucinogens, buchborn2015tolerance}.

\textbf{Mechanism:} 5-HT2A downregulation + signaling cascade depletion $\rightarrow$ reduced dendritic gain $\rightarrow$ compressed eigenmode spectrum $\rightarrow$ reduced $D_{\mathrm{eff}}$.

\subsection{Phase 3: Recanalization ($D_{\mathrm{eff}} \approx D_{\mathrm{baseline}}$ on New Landscape)}

As receptor systems recover, dimensionality returns to baseline values. However, the system does not simply revert to its prior state. The overshoot phase has exposed the system to configurations it had never occupied, potentially destabilizing maladaptive attractors and enabling reorganization onto new ones.

\textbf{The key insight:} Phase 3 involves the same dimensionality as baseline but a \emph{different attractor landscape}. The manifold the system occupies has been reshaped by the experience.

\textbf{Phenomenology:}
\begin{itemize}
\item \textbf{Lasting changes in outlook:} Altered perspectives, values, and priorities \citep{griffiths2008mystical, maclean2011mystical}
\item \textbf{Reduced depressive symptoms:} Often persisting weeks to months post-session \citep{carhart2016psilocybin, davis2021effects}
\item \textbf{Altered habits:} Reduced addictive behaviors, changed relationship patterns \citep{bogenschutz2015psilocybin, garcia2017psilocybin}
\item \textbf{Enhanced well-being:} Increased life satisfaction, meaning, and openness \citep{griffiths2006psilocybin, maclean2011mystical}
\item \textbf{Personality changes:} Measurable increases in trait openness \citep{maclean2011mystical, erritzoe2018personality}
\end{itemize}

\textbf{Neural signatures:}
\begin{itemize}
\item Normalized global entropy but altered local connectivity patterns
\item Changes in DMN-TPN (task-positive network) anticorrelation \citep{carhart2017psilocybin}
\item Increased amygdala responsiveness (not numbed, but flexible) \citep{barrett2020emotions}
\item Long-term changes in glutamate/GABA balance \citep{mason2020me}
\end{itemize}

\textbf{Duration:} Weeks to months; some changes may be permanent. The structural plasticity (increased spine density, dendritic remodeling) provides a physical substrate for persistent change \citep{ly2018psychedelics, shao2021psilocybin}.

\textbf{Mechanism:} Synaptic plasticity during high-$D_{\mathrm{eff}}$ phase (Phase 1) combined with consolidation during refractory/recovery periods produces a modified attractor landscape. The system has the same dimensionality as before but occupies different attractors.

\subsection{The Therapeutic Window: Why Timing Matters}

The three-phase model explains why integration practices and therapeutic support are critical during Phase 2 (refractory) and early Phase 3 (recanalization). During these periods:
\begin{itemize}
\item The system is actively reorganizing its attractor landscape
\item New configurations are not yet consolidated
\item Environmental inputs can bias which attractors stabilize
\item Maladaptive patterns can re-emerge if not actively addressed
\end{itemize}

This provides a mechanistic basis for the importance of ``set and setting'' extending beyond the acute phase. The recanalization window is a critical period during which therapeutic input has maximal leverage.

\section{Dimensionality Across the Lifespan}

The three-phase model connects to broader observations about cortical dimensionality across development, aging, and pathology.

\subsection{Development: High Dimensionality as Exploration}

Early development is characterized by high cortical dimensionality. Infant and child brains show:
\begin{itemize}
\item Less coherent oscillations and weaker long-range synchronization \citep{uhlhaas2009development, paus2005mapping}
\item Weaker functional connectivity hierarchies \citep{cao2014topological, supekar2009development}
\item Broader exploration of neural state space \citep{mcintosh2008development}
\item Higher neural variability and signal complexity \citep{mcintosh2008development, garrett2011age}
\end{itemize}

This high-$D_{\mathrm{eff}}$ regime enables the extensive learning required to wire up cortex appropriately. The developing brain must explore a vast space of possible connectivity patterns to find those that support adaptive behavior.

Developmental maturation involves progressive dimensional constraint---the system explores less but exploits more efficiently within learned subspaces. Myelination increases conduction velocity and synchronization \citep{paus2005mapping}; synaptic pruning removes redundant connections \citep{huttenlocher1979synaptic}; inhibitory circuit maturation sharpens selectivity \citep{hensch2005critical}. The adult brain occupies a narrower but better-optimized manifold.

\textbf{Implication:} Psychedelics may temporarily restore a ``juvenile-like'' mode of cortical function, reopening critical period-style plasticity in the adult brain \citep{ly2018psychedelics, nardou2023oxytocin}.

\subsection{Aging: Dimensional Rigidity}

Normal aging is associated with increasing cortical stiffness:
\begin{itemize}
\item Reduced neural variability and signal complexity \citep{garrett2013importance, garrett2011age}
\item Stronger, more stereotyped attractor dynamics \citep{sleimen2017age}
\item Lower effective dimensionality of spontaneous activity \citep{ponce2015resting}
\item Reduced flexibility of functional connectivity \citep{geerligs2015functional}
\end{itemize}

The aged brain occupies a narrower manifold and is less able to explore alternative configurations. Attractors that have been reinforced over decades become increasingly dominant, making change difficult.

\textbf{Implication:} The therapeutic potential of psychedelics in older populations may relate to temporary restoration of developmental-like flexibility. A single high-$D_{\mathrm{eff}}$ episode could loosen rigid attractors that have accumulated over decades. Early evidence suggests psilocybin may be particularly effective for depression in older adults \citep{agin2020older}.

\subsection{Dimensional Phenotypes: The Stability-Plasticity Continuum}

Rather than viewing neurodivergent conditions as varying degrees of pathology, the dimensionality framework suggests they represent distinct, adaptive set-points on a stability-plasticity continuum. This spectrum likely reflects an evolutionary relaxation of genetic constraints on cortical dynamics, allowing $D_{\mathrm{eff}}$ to vary more freely across individuals to meet diverse environmental demands.

\textbf{Autism Spectrum (Hyper-Stability):} May be characterized in some cases by constitutively low effective dimensionality and hyper-stable attractor dynamics \citep{dinstein2012unreliable}. In this regime, the cortex strongly ``exploits'' learned subspaces, leading to high precision, bottom-up processing fidelity, and resistance to noise. While this constrains the flexibility required for rapid social shifting, it confers exceptional advantages in systemizing and pattern recognition---a system optimized for depth over breadth.

\textbf{ADHD (Hyper-Plasticity):} Characterized by constitutively high effective dimensionality and shallow attractor basins \citep{fassbender2011dimensional, castellanos2002developmental}. In this regime, the cortex favors ``exploration'' over exploitation, maintaining a high-entropy state that allows rapid switching between tasks and novel associations. The ``distractibility'' is functionally indistinguishable from ``high-dimensional search''---a system tuned for novelty detection rather than subspace maintenance.

\textbf{The Adaptive Spectrum:} From this perspective, the human cortex has evolved to loosen the rigid biological constraints (e.g., inhibition, ephaptic coupling) that clamp dimensionality in simpler organisms. The ADHD-Autism axis represents the natural variance of this liberated parameter, ensuring the population retains both ``specialist'' (low $D_{\mathrm{eff}}$) and ``generalist'' (high $D_{\mathrm{eff}}$) phenotypes---an evolutionary bet-hedging strategy that maintains cognitive diversity.

\textbf{Acquired Dimensional Disorders:} In contrast to developmental phenotypes, some conditions represent \emph{acquired} dimensional dysregulation:
\begin{itemize}
\item \textbf{Depression:} Acquired low $D_{\mathrm{eff}}$ with excessively deep attractor basins. Rumination reflects a system ``stuck'' in self-referential loops \citep{kaiser2015large, pizzagalli2018computational}.
\item \textbf{PTSD:} Normal global $D_{\mathrm{eff}}$ but distorted local attractor structure---specific maladaptive attractors capture disproportionate state space.
\item \textbf{Addiction:} Progressive attractor deepening around drug-seeking states \citep{volkow2016brain}.
\item \textbf{Psychosis:} Acquired high $D_{\mathrm{eff}}$ with loss of attractor structure---exploration without stabilization \citep{carhart2014entropic}.
\end{itemize}

\textbf{Therapeutic implications:} This taxonomy suggests psychedelics may benefit conditions involving acquired dimensional rigidity (depression, addiction, OCD) by transiently restoring flexibility. For developmental phenotypes (autism, ADHD), the goal is not ``correction'' but understanding how dimensionality modulation interacts with baseline set-points. Emerging evidence suggests therapeutic potential for psychedelics in autism \citep{danforth2018mdma}, though such applications require sensitivity to individual differences in optimal dimensionality.

\section{Brain Rate Variability: A Clinical Biomarker}

If dimensionality is the fundamental variable that psychedelics modulate, we need a clinically accessible way to measure it. We propose \textbf{Brain Rate Variability (BRV)} as the neural analogue of heart rate variability (HRV).

\subsection{The HRV Analogy}

Heart rate variability reflects the flexibility of autonomic regulation---the system's capacity to modulate cardiac output across different demands \citep{malik1996heart, thayer2012meta}. High HRV indicates a responsive system with access to a wide dynamic range; low HRV indicates rigidity.

Mathematically, HRV can be understood as a dimensionality metric: it measures how many independent modes of variation the cardiac control system accesses. The HRV frequency bands (HF, LF, VLF) reflect different eigenmode contributions to cardiac dynamics \citep{shaffer2017overview}. High HRV corresponds to high effective dimensionality of autonomic control; low HRV corresponds to dimensional collapse onto a narrow manifold.

HRV has become a robust biomarker for:
\begin{itemize}
\item Cardiovascular health and mortality risk \citep{kleiger1987decreased, dekker1997low}
\item Depression and anxiety \citep{kemp2010heart, chalmers2014anxiety}
\item Cognitive flexibility and emotional regulation \citep{thayer2009heart}
\item Stress resilience and adaptation \citep{porges2007stress}
\end{itemize}

The parallel to cortical dimensionality is clear: HRV indexes autonomic flexibility just as $D_{\mathrm{eff}}$ indexes cortical flexibility. Both measure the system's capacity to explore a rich dynamical repertoire rather than being confined to rigid patterns. The close coupling between cortical and autonomic dynamics---mediated through the insular cortex \citep{critchley2017interoception}---suggests that psychedelic-induced changes in cortical dimensionality should propagate to autonomic control.

\subsection{Defining BRV as Metastability}

To operationalize Brain Rate Variability, we adapt the concept of \textbf{metastability} from dynamical systems theory \citep{deco2017dynamics, shanahan2010metastable}. If we treat cortical regions as coupled oscillators, the global synchronization state at time $t$ can be described by the Kuramoto order parameter $R(t)$:
\begin{equation}
R(t) = \left| \frac{1}{N} \sum_{j=1}^{N} e^{i \theta_j(t)} \right|
\end{equation}
where $N$ is the number of regions (or channels) and $\theta_j(t)$ is the instantaneous phase of region $j$ at time $t$ (derived via Hilbert transform). $R(t)$ ranges from 0 (complete desynchronization) to 1 (complete synchronization).

\textbf{Brain Rate Variability (BRV)} is defined as the variance of this synchronization over time:
\begin{equation}
\mathrm{BRV} = \frac{1}{T} \sum_{t=1}^{T} (R(t) - \bar{R})^2
\end{equation}

High BRV indicates a system that neither locks into a single fixed state (low complexity) nor remains fully incoherent (noise), but continuously traverses a rich repertoire of configurations. This mathematically formalizes the ``dynamical flexibility'' observed in the psychedelic state. In practice, BRV as metastability can be viewed as a low-dimensional surrogate for effective dimensionality: high BRV implies that many eigenmodes are intermittently recruited and released, whereas low BRV implies the system is trapped in a narrow synchronized or desynchronized regime.

The metastability interpretation connects BRV to established dynamical systems concepts \citep{cabral2014exploring, deco2017dynamics}. A system with high metastability explores many transient synchronization patterns without settling permanently into any one. While BRV is not a ``rate'' in the narrow sense of spike frequency, it mirrors HRV in function: a compact time-varying surrogate for the system's dynamical degrees of freedom.

Complementary operationalizations of BRV include:
\begin{itemize}
\item \textbf{Microstate transition rates:} How rapidly global EEG patterns switch between quasi-stable topographies \citep{michel2018eeg}
\item \textbf{Lempel-Ziv complexity:} Algorithmic complexity of the EEG time series \citep{schartner2017increased}
\item \textbf{Permutation entropy:} Information-theoretic measure of signal unpredictability \citep{bandt2002permutation}
\end{itemize}

High BRV indicates a flexible, high-dimensionality cortical state; low BRV indicates a constrained, low-dimensionality state.

\subsection{Measurement Approaches}

BRV could be measured using several approaches with varying clinical practicality:

\textbf{1. Research-grade EEG (64-256 channels):} Full spatial resolution for detailed $D_{\mathrm{eff}}$ estimation via principal component analysis of the sensor covariance matrix. Gold standard but impractical for routine clinical use.

\textbf{2. Clinical EEG (19-21 channels):} Standard 10-20 montage provides sufficient coverage for global BRV metrics. Already available in clinical settings.

\textbf{3. Consumer EEG (2-8 channels):} Devices like Muse, OpenBCI, or Emotiv provide limited but informative frontal EEG. Focus on frontal alpha dynamics, which show strong psychedelic effects \citep{carhart2016neural}.

\textbf{4. Eye tracking:} Pupil diameter variability and microsaccade patterns serve as proxies for cortical state variability via the superior colliculus and locus coeruleus pathways \citep{joshi2016pupil, engbert2003microsaccades}. Pupil diameter reflects noradrenergic/cholinergic tone, which covaries with cortical dimensionality.

\textbf{5. Combined approaches:} A ``BRV glasses'' device with frontal electrodes (2-4 channels) and integrated eye tracking could provide continuous, naturalistic measurement. This would enable:
\begin{itemize}
\item Real-time BRV monitoring throughout psychedelic sessions
\item Outpatient tracking during refractory and recanalization phases
\item Baseline assessment for risk stratification
\item Long-term monitoring of treatment effects
\end{itemize}

Such a device is technically feasible with current hardware and would fill a significant gap in psychedelic research and therapy.

\section{Predictions and Tests}

The dimensionality modulation framework generates specific, testable predictions:

\subsection{Acute Phase Predictions}

\textbf{P1:} EEG-derived $D_{\mathrm{eff}}$ (via participation ratio or related measures) should peak 60--120 minutes post-administration, correlating with subjective intensity ratings.

\textbf{P2:} The $D_{\mathrm{eff}}$ increase should be dose-dependent, with perceptual threshold effects corresponding to dimensionality expansion threshold.

\textbf{P3:} 5-HT2A antagonist pre-treatment (ketanserin) should block the dimensionality increase, not just subjective effects.

\textbf{P4:} Individuals with higher baseline $D_{\mathrm{eff}}$ may require higher doses to achieve equivalent expansion, predicting ceiling effects and individual dose-response variation.

\textbf{P5:} The dimensionality increase should be detectable across multiple measurement modalities (EEG, fMRI, pupillometry) with correlated magnitudes.

\subsection{Refractory Phase Predictions}

\textbf{P6:} $D_{\mathrm{eff}}$ should drop below baseline 12--48 hours post-experience, correlating with subjective fatigue and tolerance.

\textbf{P7:} This refractory period should correlate with 5-HT2A receptor occupancy recovery measured by PET imaging.

\textbf{P8:} Repeated dosing within the refractory window should produce attenuated $D_{\mathrm{eff}}$ increase (pharmacological tolerance).

\textbf{P9:} HRV and BRV should show correlated refractory dynamics, reflecting coupled autonomic-cortical dimensionality modulation.

\subsection{Recanalization Phase Predictions}

\textbf{P10:} Return to baseline $D_{\mathrm{eff}}$ should be accompanied by altered functional connectivity patterns (same dimensionality, different manifold).

\textbf{P11:} The magnitude of acute $D_{\mathrm{eff}}$ increase should predict the extent of connectivity reorganization, controlling for subjective experience metrics.

\textbf{P12:} Structural imaging should show spine density changes correlated with functional connectivity reorganization.

\subsection{Therapeutic Predictions}

\textbf{P13:} Therapeutic response should correlate with the magnitude of acute $D_{\mathrm{eff}}$ increase, controlling for mystical experience scores.

\textbf{P14:} Integration practices during recanalization should enhance outcomes by stabilizing beneficial attractor reorganization.

\textbf{P15:} Patients with excessively low baseline $D_{\mathrm{eff}}$ (severe depression, rigid patterns) should show larger therapeutic responses than those with normal baseline dimensionality.

\textbf{P16:} Patients with high baseline $D_{\mathrm{eff}}$ or unstable dynamics should be at higher risk for adverse outcomes (anxiety, psychotic features).

\section{Empirical Validation: LSD fMRI Data}

To directly test the core prediction that psychedelics increase effective dimensionality, we reanalyzed the Carhart-Harris LSD dataset (OpenNeuro ds003059) \citep{carhart2016neural}. This dataset contains resting-state fMRI from 15 healthy participants under both LSD (75$\mu$g IV) and placebo in a within-subjects crossover design.

\subsection{Methods}

We extracted ROI time series using the Schaefer 200-parcel atlas \citep{schaefer2018local} and computed $D_{\mathrm{eff}}$ via participation ratio of the covariance matrix eigenvalues (Equation 1). Analysis used the first available run from each session to ensure temporal consistency.

\subsection{Results}

Table \ref{tab:deff_results} shows individual subject results.

\begin{table}[htbp]
\centering
\caption{\textbf{Effective Dimensionality Under LSD vs. Placebo.} Individual subject $D_{\mathrm{eff}}$ values computed from resting-state fMRI using 200-parcel Schaefer atlas.}
\label{tab:deff_results}
\begin{tabular}{lccc}
\toprule
Subject & LSD & Placebo & Ratio \\
\midrule
sub-001 & 9.52 & 9.73 & 0.98 \\
sub-002 & 11.36 & 10.29 & 1.10 \\
sub-003 & 9.61 & 7.99 & 1.20 \\
sub-004 & 12.44 & 10.67 & 1.17 \\
sub-006 & 12.63 & 10.60 & 1.19 \\
sub-009 & 11.19 & 10.44 & 1.07 \\
sub-010 & 13.54 & 11.12 & 1.22 \\
sub-011 & 12.55 & 11.82 & 1.06 \\
sub-012 & 11.48 & 8.39 & 1.37 \\
sub-013 & 12.14 & 10.67 & 1.14 \\
sub-015 & 12.51 & 8.27 & 1.51 \\
sub-017 & 6.99 & 8.91 & 0.78 \\
sub-018 & 11.49 & 12.34 & 0.93 \\
sub-019 & 9.04 & 10.89 & 0.83 \\
sub-020 & 9.48 & 10.72 & 0.88 \\
\midrule
\textbf{Mean} & \textbf{11.06} & \textbf{10.19} & \textbf{1.09} \\
\textbf{SD} & 1.71 & 1.24 & --- \\
\bottomrule
\end{tabular}
\end{table}

Group analysis revealed:
\begin{itemize}
\item \textbf{LSD:} $D_{\mathrm{eff}} = 11.06 \pm 1.71$ (mean $\pm$ SD)
\item \textbf{Placebo:} $D_{\mathrm{eff}} = 10.19 \pm 1.24$
\item \textbf{Difference:} $+0.87$ ($+8.6\%$ increase under LSD)
\item \textbf{Paired t-test:} $t = 1.88$, $p = 0.08$
\item \textbf{Effect size:} Cohen's $d = 0.50$ (medium effect)
\item \textbf{Individual effects:} 10/15 subjects (67\%) showed higher $D_{\mathrm{eff}}$ under LSD
\end{itemize}

\subsection{Discussion}

These results provide direct empirical support for the dimensionality modulation hypothesis. Despite modest sample size (N=15) and the inherent limitations of BOLD fMRI for capturing fast neural dynamics, we observe an 8.6\% increase in effective dimensionality with a medium effect size.

Several factors may attenuate the observed effect:
\begin{enumerate}
\item \textbf{Temporal resolution:} BOLD fMRI (TR = 2s) cannot capture the millisecond-scale dynamics where dimensionality changes may be most pronounced
\item \textbf{Parcellation:} 200 ROIs represent a compressed representation of cortical dynamics; finer-grained analyses might reveal larger effects
\item \textbf{Timing:} Scans were acquired during peak drug effects, not necessarily at maximal dimensionality
\item \textbf{Individual variability:} The 5 subjects showing decreased $D_{\mathrm{eff}}$ (sub-001, -017, -018, -019, -020) may reflect responder heterogeneity, timing differences, or methodological factors
\end{enumerate}

\textbf{The temporal scale gap.} A primary limitation is the temporal resolution mismatch between our proposed mechanism and the empirical validation. The theoretical framework relies on dendritic calcium spikes and eigenmode expansion occurring at millisecond timescales, while BOLD fMRI has a temporal resolution of approximately 2 seconds. However, hemodynamic signals act as a low-pass filter of neural activity. Recent work on cross-frequency coupling suggests that changes in high-frequency neural dimensionality (e.g., gamma/alpha desynchronization) propagate to low-frequency hemodynamic fluctuations. We propose that the expanded $D_{\mathrm{eff}}$ observed in BOLD fMRI is consistent with a macroscopic echo of the underlying microscopic expansion. While fMRI cannot resolve individual dendritic events, it successfully captures the resulting reorganization of the global attractor landscape. Future studies employing MEG or simultaneous EEG-fMRI will be necessary to fully characterize the transfer function between dendritic gain dynamics and whole-brain functional dimensionality.

Notably, two subjects (sub-012 and sub-015) showed particularly large effects (37\% and 51\% increases), suggesting substantial individual variability in dimensionality response. This variability itself is predicted by the framework: individuals with already-high baseline $D_{\mathrm{eff}}$ may show ceiling effects, while those with lower baseline may show larger expansion.

The medium effect size ($d = 0.50$) is comparable to or larger than many established psychedelic neural signatures and provides quantitative support for the central claim that LSD expands the effective dimensionality of cortical dynamics.

\subsection{Cross-Compound Replication: Psilocybin}

To test whether dimensionality expansion generalizes across classical psychedelics, we analyzed data from the Siegel et al. psilocybin precision functional mapping study (OpenNeuro ds006072) \citep{siegel2025psilocybin}. This dataset employs a within-subjects crossover design comparing psilocybin (25mg oral) versus methylphenidate (40mg, active control) in 7 healthy participants with dense baseline imaging (5+ sessions per subject).

We computed $D_{\mathrm{eff}}$ from preprocessed CIFTI dense time series data (91,206 grayordinates, subsampled to 5,000 for computational efficiency) using the same participation ratio metric. Results from the initial subjects show:

\begin{itemize}
\item \textbf{Baseline:} $D_{\mathrm{eff}} = 56.6 \pm 5.7$ (averaged across baseline sessions)
\item \textbf{Acute drug sessions:} $D_{\mathrm{eff}} = 67.5 \pm 18.3$
\item \textbf{Change:} $+19.2\%$ increase in effective dimensionality
\item \textbf{Effect size:} Cohen's $d = 0.80$ (large)
\end{itemize}

Critically, the crossover design allows separation of psilocybin from methylphenidate sessions using MEQ (Mystical Experience Questionnaire) scores. Sessions with high MEQ scores (psilocybin) showed dramatically different $D_{\mathrm{eff}}$ than sessions with near-zero MEQ scores (methylphenidate). In Subject P1 (the only participant with complete pharmacological dissociation data at time of analysis), psilocybin produced $+25.2\%$ $D_{\mathrm{eff}}$ expansion (MEQ Mystical = 4.37/5), while methylphenidate produced $-15.7\%$ $D_{\mathrm{eff}}$ \emph{compression} (MEQ Mystical = 0.0/5). This bidirectional dissociation---psilocybin expands, methylphenidate contracts---provides strong evidence that dimensionality modulation is specific to 5-HT2A agonism rather than generic arousal or task engagement.

\textbf{Phase 3 validation.} The Siegel dataset includes longitudinal follow-up sessions (``After'' scans) acquired days to weeks post-drug, enabling direct testing of the recanalization hypothesis. Preliminary analysis of follow-up sessions shows $D_{\mathrm{eff}} = 58.2 \pm 4.6$, which returns to near-baseline levels (compare to baseline $56.6 \pm 5.7$, acute $67.5 \pm 18.3$). This pattern---acute expansion followed by return to baseline dimensionality---is precisely what the three-phase model predicts. The therapeutic reorganization occurs not through permanently elevated $D_{\mathrm{eff}}$, but through the exploration enabled during the overshoot phase, consolidated during recanalization onto a modified attractor landscape.

These preliminary results provide cross-compound validation of the dimensionality hypothesis. The larger effect size in the psilocybin data (+19.2\%, $d = 0.80$) compared to LSD (+8.6\%, $d = 0.50$) may reflect methodological differences (higher spatial resolution of CIFTI data, denser baseline sampling), but may also capture genuine pharmacological differences with profound phenomenological correlates. Future work should explicitly correlate the magnitude of $\Delta D_{\mathrm{eff}}$ with subjective intensity metrics (e.g., MEQ total scores) to establish a direct psychometric link between dimensionality expansion and mystical experience.

\subsection{Geometric versus Organic: Eigenmode Structure and Subjective Experience}

An intriguing difference emerged when comparing dimensionality changes across compounds: psilocybin produced a substantially larger increase in $D_{\mathrm{eff}}$ (+19.2\%) than LSD (+8.6\%). This aligns with long-standing phenomenological distinctions between the two drugs. LSD experiences are frequently described as ``geometric''---structured lattices, grids, and fractal symmetries dominate the visual field. Psilocybin imagery, by contrast, is often characterized as ``organic''---fluid, earthy, boundary-dissolving, with morphing textures and entangled forms.

In dynamical-systems terms, geometric structure corresponds to the amplification of a small set of low-frequency, symmetry-preserving eigenmodes. The visual cortex possesses intrinsic Gabor-like filter structure; LSD may primarily boost the gain on these built-in geometric modes, producing strong structured visuals without dramatically expanding the total number of active degrees of freedom. Organic complexity, by contrast, implies the recruitment of a larger and less structured set of high-frequency modes---activity spilling into directions that are normally suppressed as noise.

This interpretation suggests that psilocybin's larger dimensionality increase reflects a more pervasive destabilization of the cortical energy landscape, enabling widespread exploration of latent oscillatory modes. The pharmacological basis may involve psilocin's faster binding dynamics and broader receptor engagement compared to LSD's prolonged, tight 5-HT2A binding. LSD ``locks in'' to a specific gain state, amplifying structure; psilocybin destabilizes more broadly, liberating chaos.

To test this prediction, we computed the spectral centroid of the cortical eigenspectrum---a measure of the ``center of mass'' of the energy distribution across eigenmodes. Higher centroid values indicate greater recruitment of high-frequency modes.

The results strongly support the geometric-versus-organic distinction:
\begin{itemize}
\item \textbf{LSD:} Spectral centroid increased by $+10.0\%$ relative to placebo ($p = 0.0008$, $N = 15$)
\item \textbf{Psilocybin:} Spectral centroid increased by $+18.6\%$ relative to baseline (high-density case study: $N = 1$ subject with 5 baseline sessions vs 2 drug sessions, providing within-subject replication)
\end{itemize}

The psilocybin spectral shift is nearly twice that of LSD, mirroring the ratio of their dimensionality increases ($+19.2\%$ vs $+8.6\%$). This quantifies a fundamental distinction: LSD produces \emph{structured expansion}---dimensionality increases moderately while energy remains largely constrained to low-frequency geometric modes. Psilocybin produces \emph{chaotic expansion}---dimensionality increases dramatically as energy spills into high-frequency modes that are normally suppressed.

This ``spectral tilt'' provides a biophysical basis for qualitative phenomenological reports. The crystalline symmetries of LSD reflect amplification of the cortex's intrinsic geometric structure; the fluid chaos of psilocybin reflects escape from that structure into normally-inaccessible high-frequency dynamics.

\section{Implications and Future Directions}

\subsection{Precision Dosing}

If dimensionality is the therapeutic target, real-time BRV monitoring could enable precision dosing: titrating administration to achieve a target $D_{\mathrm{eff}}$ increase rather than a fixed milligram dose. This approach could account for individual differences in:
\begin{itemize}
\item Receptor density and distribution
\item Metabolic rate (CYP2D6 polymorphisms for psilocybin)
\item Baseline cortical state
\item Prior psychedelic experience
\item Current medication effects
\end{itemize}

Adaptive dosing protocols could use BRV feedback to adjust administration rate during IV infusion or guide supplementary dosing during oral sessions.

\subsection{Combination Therapies}

The three-phase model suggests opportunities for combination approaches:

\textbf{Phase 1 modulation:} Agents that extend or deepen the overshoot phase could enhance therapeutic exploration. Possibilities include:
\begin{itemize}
\item MAO inhibitors (extending duration, as in ayahuasca)
\item Agents that reduce 5-HT2A internalization
\item NMDA modulators that enhance plasticity
\end{itemize}

\textbf{Phase 2 modulation:} Agents that shorten the refractory phase could enable more frequent dosing. This requires caution---the refractory period may serve protective functions.

\textbf{Phase 3 optimization:} Interventions that enhance recanalization could improve outcomes:
\begin{itemize}
\item Structured integration protocols
\item Targeted psychotherapy during the plasticity window
\item Physical exercise (which enhances neural plasticity)
\item Sleep optimization (critical for consolidation)
\end{itemize}

\subsection{Non-Psychedelic Dimensionality Modulation}

If dimensionality is the key therapeutic variable, other interventions that modulate $D_{\mathrm{eff}}$ might produce similar benefits without requiring the intense subjective experience:

\textbf{Brain stimulation:} Transcranial alternating current stimulation (tACS) or transcranial magnetic stimulation (TMS) targeting eigenmode expansion. Preliminary work suggests tACS can modulate cortical complexity \citep{reinhart2017working}.

\textbf{Neurofeedback:} Training protocols that reward high BRV states, gradually expanding the accessible dimensionality through operant conditioning.

\textbf{Meditation:} Contemplative practices alter cortical dynamics and may produce dimensionality modulation effects \citep{lutz2004long, tang2015brain}. Advanced meditators show altered baseline $D_{\mathrm{eff}}$ and enhanced flexibility.

\textbf{Other pharmacology:} Compounds affecting dendritic gain through non-5-HT2A mechanisms (e.g., NMDA modulators, specific ion channel modulators) might produce dimensionality expansion with different subjective profiles.

\subsection{Risk Stratification}

The framework clarifies risks and contraindications:

\textbf{High-risk populations:}
\begin{itemize}
\item Personal or family history of psychotic disorders (already high/unstable $D_{\mathrm{eff}}$)
\item Severe anxiety disorders (may not tolerate dimensionality expansion)
\item Some ADHD presentations (already excessive $D_{\mathrm{eff}}$)
\item Current manic or hypomanic states
\item Unstable personality disorders during acute crisis
\end{itemize}

\textbf{Lower-risk populations:}
\begin{itemize}
\item Treatment-resistant depression with rigid patterns
\item Stable anxiety with good emotional regulation capacity
\item Addiction in motivated individuals
\item Existential distress in terminal illness
\end{itemize}

Baseline BRV/HRV assessment could inform risk stratification, identifying individuals with dimensionality profiles that predict positive vs. adverse responses.

\subsection{Broader Implications}

The dimensionality framework suggests that psychedelics are not pharmacologically unique---they are revealing a general principle of neural computation. Dimensionality is the computational currency of cortical flexibility. Systems with appropriate dimensionality can learn, adapt, and maintain health; systems with too little dimensionality become rigid and pathological; systems with too much become chaotic and dysfunctional.

This perspective reframes psychedelic therapy from ``chemical intervention'' to ``dimensionality modulation.'' The specific molecule matters less than the dimensionality dynamics it produces. Future developments might identify optimal dimensionality trajectories for different conditions and optimize interventions to achieve them, whether through pharmacology, stimulation, behavior, or some combination.

\section{Conclusion}

We have proposed that classical psychedelics are fundamentally dimensionality modulators---they expand and then compress the effective dimensionality of cortical dynamics, enabling exploration of normally inaccessible configurations and subsequent reorganization onto modified attractor landscapes.

This framework unifies observations across scales:
\begin{itemize}
\item \textbf{Molecular:} 5-HT2A activation $\rightarrow$ dendritic gain amplification $\rightarrow$ eigenmode threshold reduction
\item \textbf{Cellular:} Enhanced dendritic spikes, facilitated EPSPs, structural plasticity
\item \textbf{Circuit:} Desynchronization, decoupling, reduced ephaptic constraint
\item \textbf{Systems:} DMN dissolution, altered functional connectivity, flattened hierarchy
\item \textbf{Computational:} Expanded reservoir capacity, exploration of off-manifold states
\item \textbf{Phenomenological:} Perceptual intensification, ego dissolution, novel associations
\item \textbf{Therapeutic:} Destabilization of maladaptive attractors, recanalization onto healthier patterns
\end{itemize}

The three-phase model (overshoot $\rightarrow$ refractory $\rightarrow$ recanalization) provides a temporal structure for understanding both acute effects and lasting plasticity. Brain Rate Variability offers a path toward clinical measurement of the dimensionality dynamics that underlie therapeutic outcomes.

Dimensionality is not merely a mathematical abstraction---it is the computational currency of cortical flexibility. Psychedelics are powerful therapeutic tools precisely because they modulate this fundamental variable, enabling the brain to temporarily escape its learned constraints and reorganize its computational landscape. Understanding this principle opens new avenues for precision dosing, risk stratification, combination therapies, and non-psychedelic interventions targeting the same underlying mechanism.

\section*{Acknowledgements}

This work was supported by the University of Sydney.

\section*{Author Contributions}

I.T. conceived the theoretical framework, designed and performed all analyses, and wrote the manuscript.

\section*{Competing Interests}

The author declares no competing interests.

\section*{Data Availability}

All neuroimaging data analyzed in this study are publicly available from OpenNeuro: the LSD dataset (ds003059) at \url{https://openneuro.org/datasets/ds003059} and the psilocybin precision functional mapping dataset (ds006072) at \url{https://openneuro.org/datasets/ds006072}. The original studies obtained ethics approval and informed consent as described in the respective publications \citep{carhart2016neural, siegel2025psilocybin}.

\section*{Code Availability}

Analysis code for computing effective dimensionality and spectral centroid from CIFTI/NIfTI data is available at \url{https://github.com/todd866/lsd-dimensionality}.

\bibliographystyle{unsrtnat}
\bibliography{references}

\end{document}
