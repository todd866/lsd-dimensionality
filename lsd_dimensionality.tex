\documentclass[11pt,a4paper]{article}
\usepackage[utf8]{inputenc}
\usepackage{amsmath,amssymb}
\usepackage{graphicx}
\usepackage[margin=1in]{geometry}
\usepackage{hyperref}
\usepackage{natbib}
\usepackage{booktabs}

\title{Timescale-Dependent Cortical Dimensionality: \\
\large Psychedelics Desynchronize Fast Oscillations but Spare the Slow Hemodynamic Manifold}

\author{Ian Todd\\
\texttt{itod2305@uni.sydney.edu.au}}

\date{\today}

\begin{document}

\maketitle

\begin{abstract}
The Entropic Brain hypothesis posits that psychedelics increase the richness of cortical states---in dynamical terms, an expansion of effective dimensionality. We tested this prediction across two neuroimaging modalities with distinct temporal resolutions. In MEG (136 sessions across four compounds), classical psychedelics produced significant oscillatory desynchronization (psilocybin: $-15\%$ coherence, $p = 0.003$, $d = -0.78$), while ketamine showed no effect---consistent with 5-HT2A-specific mechanism. Conversely, fMRI analysis ($N = 7$, 124 sessions) revealed no change in hemodynamic dimensionality ($-5.7\%$, $p = 0.47$), though the spectral centroid shifted toward higher modes ($+16.8\%$, $p = 0.09$, $d = 0.82$). These results demonstrate a critical dissociation: 5-HT2A agonism disrupts the coherence of fast neural oscillations, but this entropy does not propagate to the slow metabolic manifold measured by BOLD. We conclude that ``dimensionality expansion'' is a timescale-specific phenomenon concentrated at 10--100 Hz, not 0.01--0.1 Hz. Future accounts of psychedelic mechanism must reconcile how increased information diversity at the millisecond scale integrates into stable hemodynamic patterns.
\end{abstract}

\section{Introduction}

If the cortex operates as a system of coupled neural oscillators \citep{miller2018working, buzsaki2006rhythms}, it is inherently high-dimensional. The state space includes the phases and amplitudes of oscillators across frequencies and regions. A natural question arises: do psychedelics change this dimensionality?

By ``dimensionality'' we mean the number of independent modes of variation---how many degrees of freedom are actively being used. A system with 1000 oscillators might have effective dimensionality of 10 (if they're all synchronized) or 500 (if they're largely independent).

The question matters because dimensionality determines computational capacity. Low-dimensional systems are constrained to a narrow manifold; high-dimensional systems can explore more configurations. If psychedelics increase dimensionality, this could explain both their acute effects (access to unusual states) and their therapeutic potential (escape from rigid attractor patterns).

\subsection{The Prediction}

Classical psychedelics are 5-HT2A agonists. The 5-HT2A receptor is concentrated on layer 5 pyramidal neurons, particularly in apical dendrites \citep{nichols2016psychedelics}. Activation increases dendritic gain---the same input produces larger postsynaptic responses.

In a coupled oscillator system, this predicts desynchronization:
\begin{itemize}
\item Higher gain $\rightarrow$ neurons respond more to local inputs
\item Local inputs include noise and weak signals normally below threshold
\item This disrupts the coherent oscillations that synchronize populations
\item More independent oscillators $\rightarrow$ higher effective dimensionality
\end{itemize}

The prediction is qualitative: psychedelics should \emph{increase} dimensionality. But a critical question remains: at what timescale?

\section{Methods}

\subsection{MEG Analysis}

We analyzed MEG data from 136 sessions across four compounds: psilocybin ($N=40$), LSD ($N=30$), ketamine ($N=36$), and tiagabine ($N=30$). Data were from publicly available datasets \citep{muthukumaraswamy2013broadband, carhart2016neural}.

MEG detects synchronous postsynaptic currents with millisecond resolution. We computed the participation ratio of sensor covariance as a proxy for oscillatory coherence structure. Lower values indicate more distributed, desynchronized activity.

\subsection{fMRI Analysis}

We analyzed fMRI data from the Siegel psilocybin precision functional mapping study (OpenNeuro ds006072): 7 subjects, 124 sessions total, with dense baseline imaging \citep{siegel2025psilocybin}.

fMRI measures hemodynamic responses with $\sim$1 second resolution, reflecting metabolic demand integrated over $\sim$10 seconds. We computed two metrics from CIFTI grayordinate time series (91,206 voxels):

\textbf{Effective dimensionality} via participation ratio:
\begin{equation}
D_{\mathrm{eff}} = \frac{\left(\sum_i \lambda_i\right)^2}{\sum_i \lambda_i^2}
\end{equation}

\textbf{Spectral centroid}---the center of mass of the eigenspectrum:
\begin{equation}
C = \frac{\sum_i i \cdot \lambda_i}{\sum_i \lambda_i}
\end{equation}

The spectral centroid measures where variance is concentrated. A shift toward higher modes indicates redistribution of variance from dominant low-frequency patterns to finer-grained activity, even if total dimensionality is unchanged.

\section{Results}

\subsection{MEG: Fast Oscillatory Desynchronization}

\begin{table}[h]
\centering
\caption{MEG coherence changes by compound}
\begin{tabular}{lccc}
\toprule
Compound & Change & $p$ & Cohen's $d$ \\
\midrule
Psilocybin & $-15.0\%$ & 0.003 & $-0.78$ \\
LSD & $-13.4\%$ & 0.082 & $-0.50$ \\
Ketamine & $+5.7\%$ & 0.290 & $+0.26$ \\
Tiagabine & $+10.8\%$ & 0.307 & $+0.28$ \\
\bottomrule
\end{tabular}
\end{table}

Classical psychedelics (5-HT2A agonists) produce significant desynchronization. Ketamine (NMDA antagonist) does not. This dissociation is consistent with the prediction that 5-HT2A-mediated gain increase specifically disrupts oscillatory coherence.

\subsection{fMRI: Stable Hemodynamic Manifold}

\textbf{Effective dimensionality:}
\begin{itemize}
\item Baseline: $D_{\mathrm{eff}} = 51.5 \pm 7.4$
\item Drug: $D_{\mathrm{eff}} = 49.0 \pm 12.4$
\item Change: $-5.7\%$ (non-significant)
\item Statistics: $t(6) = -0.78$, $p = 0.47$, $d = -0.32$
\end{itemize}

\textbf{Spectral centroid:}
\begin{itemize}
\item Baseline: $C = 80.8 \pm 9.5$
\item Drug: $C = 93.7 \pm 16.3$
\item Change: $+16.8\%$
\item Statistics: $t(6) = 2.01$, $p = 0.09$, $d = 0.82$
\end{itemize}

The participation ratio shows no significant change. However, the spectral centroid shifts substantially toward higher modes, with a large effect size ($d = 0.82$) that trends toward significance despite the small sample. This indicates that while the \emph{count} of dimensions is stable, the \emph{distribution} of variance across modes changes---energy moves from the first few dominant patterns toward finer-grained activity.

\section{Discussion}

\subsection{A Timescale Dissociation}

The MEG and fMRI results are not contradictory---they reveal a timescale-dependent phenomenon. Psychedelics desynchronize fast oscillatory dynamics (10--100 Hz), increasing the independence of neural populations at the millisecond scale. But this entropy does not propagate to the slow hemodynamic manifold (0.01--0.1 Hz).

Why might this be? Several mechanisms could filter out fast dimensionality expansion:

\textbf{Temporal smoothing.} BOLD integrates metabolic demand over $\sim$10 seconds. Rapid fluctuations in neural configuration may average out, preserving only slow covariance structure.

\textbf{Metabolic homeostasis.} Even if moment-to-moment neural patterns become more diverse, total metabolic demand may be regulated. The brain might explore more configurations while consuming similar energy, leaving the hemodynamic signature unchanged.

\textbf{Network-level constraints.} Large-scale functional networks may impose structure on BOLD dynamics that fast oscillatory changes cannot escape. The ``slow manifold'' may be more constrained by anatomy and vasculature than by neural dynamics.

\subsection{The Spectral Centroid Finding}

The shift in spectral centroid ($+16.8\%$, $d = 0.82$) suggests the hemodynamic manifold is not entirely invariant. While total dimensionality doesn't change, variance redistributes toward higher modes. This is consistent with a subtle ``flattening'' of the eigenspectrum---the first few principal components explain slightly less variance, with energy spreading to finer-grained patterns.

This finding partially rescues the fMRI data from being purely null. It suggests that even at slow timescales, psychedelics alter the \emph{structure} of cortical dynamics, even if they don't change its \emph{dimensionality} as measured by participation ratio.

\subsection{Implications for the Entropic Brain Hypothesis}

The Entropic Brain hypothesis \citep{carhart2014entropic} proposes that psychedelics increase the entropy of spontaneous brain activity. Our results suggest this is true---but only at fast timescales. The ``expansion of consciousness'' appears to be a high-frequency phenomenon.

This has implications for therapeutic mechanisms. If psychedelics enable escape from rigid attractor patterns, this flexibility may operate at the oscillatory level (enabling rapid state transitions) rather than the metabolic level (stable network configurations). Therapeutic benefit might arise from increased \emph{exploration rate} rather than increased \emph{state space size}.

\subsection{Limitations}

\textbf{Sample size.} The fMRI analysis includes only 7 subjects. While precision mapping provides many sessions per subject, between-subject variance limits statistical power for detecting population effects.

\textbf{No GSR comparison.} The analyzed data did not include global signal regression. If psychedelics increase global connectivity, this could mask local dimensionality increases.

\textbf{Single fMRI metric.} Participation ratio is one of many possible dimensionality measures. Alternative metrics (e.g., correlation dimension, Kaplan-Yorke dimension) might reveal effects we missed.

\section{Conclusion}

We demonstrate that psychedelic ``dimensionality expansion'' is timescale-specific:

\begin{enumerate}
\item \textbf{Fast oscillations (MEG):} 5-HT2A agonists produce significant desynchronization, increasing the independence of neural populations. This effect is mechanism-specific---ketamine shows no such change.

\item \textbf{Slow hemodynamics (fMRI):} Participation ratio is unchanged, but spectral centroid shifts toward higher modes. The slow manifold is altered in structure but not in dimensionality.

\item \textbf{Timescale dissociation:} The entropy induced by psychedelics is concentrated at fast timescales and does not propagate to slow metabolic dynamics.
\end{enumerate}

Future work should examine intermediate timescales (EEG source localization, MEG-fMRI fusion) to determine where dimensionality expansion attenuates. The therapeutic mechanisms of psychedelics may depend critically on this timescale structure.

\section*{Acknowledgements}

The author thanks the original investigators who made their datasets publicly available on OpenNeuro.

\section*{Data Availability}

All data analyzed are publicly available: LSD dataset (ds003059) and psilocybin dataset (ds006072) at \url{https://openneuro.org}.

\section*{Code Availability}

Analysis code is available at \url{https://github.com/todd866/lsd-dimensionality}.

\bibliographystyle{unsrtnat}
\bibliography{references}

\end{document}
