\documentclass[11pt,a4paper]{letter}
\usepackage[margin=1in]{geometry}
\usepackage{hyperref}

\signature{Ian Todd\\Sydney Medical School\\University of Sydney}
\address{Sydney Medical School\\University of Sydney\\Sydney, NSW, Australia\\itod2305@uni.sydney.edu.au}

\begin{document}

\begin{letter}{Editor-in-Chief\\Translational Psychiatry}

\opening{Dear Editor,}

We submit our manuscript ``\textbf{Psychedelics as Dimensionality Modulators: A Cortical Reservoir Theory of Serotonergic Plasticity}'' for consideration as an Article in \textit{Translational Psychiatry}.

\textbf{Clinical significance.} As psychedelic-assisted therapy enters mainstream psychiatry, clinicians face a critical challenge: how do we quantify the neural mechanisms that distinguish therapeutic response from mere drug effect? Our work provides a direct answer: MEG-derived oscillatory coherence serves as a real-time, mechanism-specific biomarker for ``psychedelic depth.''

\textbf{Key findings with immediate translational relevance:}
\begin{enumerate}
\item \textbf{Mechanism-specific biomarker.} Analysis of 136 MEG sessions across four compounds reveals a striking dissociation: classical psychedelics (psilocybin, LSD) produce significant oscillatory desynchronization (psilocybin: $-$15\%, $p = 0.003$, $d = -0.78$), while ketamine shows no effect ($p = 0.29$). This specificity---psychedelics desynchronize, dissociatives do not---provides a neural signature that distinguishes therapeutic mechanisms at the neurophysiological level.

\item \textbf{Clinical decision support.} The MEG biomarker enables three immediate clinical applications: (i) real-time monitoring during therapeutic sessions, (ii) precision dosing based on neural response rather than fixed milligrams, and (iii) patient selection by identifying likely responders through baseline cortical flexibility.

\item \textbf{Mechanistic framework for treatment optimization.} Our three-phase model (overshoot $\rightarrow$ refractory $\rightarrow$ recanalization) provides a principled basis for integration timing, session spacing, and adjunctive interventions during the critical plasticity window.
\end{enumerate}

\textbf{Why Translational Psychiatry.} This work bridges basic neuroscience (eigenmode dynamics, reservoir computing) with clinical application (biomarkers, dosing protocols). The dimensionality framework moves psychedelic research from ``entropy increases'' to quantitative, testable predictions about therapeutic mechanisms. \textit{Translational Psychiatry}'s readership---spanning psychiatrists, neuroscientists, and clinical researchers---is ideally positioned to evaluate and implement these findings.

\textbf{Scope and novelty.} While dimensionality metrics have been applied descriptively to psychedelic neuroimaging, our contribution is fundamentally different: we propose that effective dimensionality is not merely a correlate but the \textit{computational function} of psychedelic therapy. The MEG validation provides the first compound-comparison evidence that oscillatory desynchronization is mechanism-specific, with direct implications for distinguishing psychedelic-assisted from ketamine-assisted therapy.

All data analyzed are publicly available (OpenNeuro ds003059, ds006072; Muthukumaraswamy MEG repository). Code is available at \url{https://github.com/todd866/lsd-dimensionality}. The author declares no competing interests.

\closing{Respectfully submitted,}

\end{letter}
\end{document}
